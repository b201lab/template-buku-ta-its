% Judul dokumen
\title{Buku Tugas Akhir ITS}
\author{Musk, Elon Reeve}

% Pengaturan ukuran teks dan bentuk halaman dua sisi
\documentclass[10pt,twoside]{report}

% Pengaturan ukuran halaman dan margin
\usepackage[a5paper,top=25mm,left=25mm,right=20mm,bottom=25mm]{geometry}

% Pengaturan ukuran spasi
\usepackage[singlespacing]{setspace}

% Pengaturan format bahasa
\usepackage[indonesian]{babel}

% Pengaturan detail pada file PDF
\usepackage[pdfauthor={\@author},bookmarksnumbered,pdfborder={0 0 0}]{hyperref}

% Pengaturan jenis karakter
\usepackage[utf8]{inputenc}

% Pengaturan pewarnaan
\usepackage[table,xcdraw]{xcolor}

% Pengaturan kutipan artikel
\usepackage[numbers]{natbib}

% Package lainnya
\usepackage{changepage}
\usepackage{enumitem}
\usepackage{eso-pic}
\usepackage{etoolbox}
\usepackage{graphicx}
\usepackage{lipsum}
\usepackage{lmodern}
\usepackage{longtable}
\usepackage{tabularx}
\usepackage{wrapfig}

% Definisi untuk "Hati ini sengaja dikosongkan"
\patchcmd{\cleardoublepage}{\hbox{}}{
  \thispagestyle{empty}
  \vspace*{\fill}
  \begin{center}\textit{[Halaman ini sengaja dikosongkan]}\end{center}
  \vfill}{}{}

% Pengaturan penomoran halaman
\usepackage{fancyhdr}
\fancyhf{}
\renewcommand{\headrulewidth}{0pt}
\pagestyle{fancy}
\fancyfoot[CE,CO]{\thepage}
\patchcmd{\chapter}{plain}{fancy}{}{}
\patchcmd{\chapter}{empty}{plain}{}{}

% Pengaturan format judul bab
\usepackage{titlesec}
\titleformat{\chapter}[display]{\bfseries\Large}{BAB \centering\Roman{chapter}}{0ex}{\vspace{0ex}\centering}
\titleformat{\section}{\bfseries\large}{\MakeUppercase{\thesection}}{1ex}{\vspace{1ex}}
\titleformat{\subsection}{\bfseries\large}{\MakeUppercase{\thesubsection}}{1ex}{}
\titleformat{\subsubsection}{\bfseries\large}{\MakeUppercase{\thesubsubsection}}{1ex}{}
\titlespacing{\chapter}{0ex}{0ex}{4ex}
\titlespacing{\section}{0ex}{1ex}{0ex}
\titlespacing{\subsection}{0ex}{0.5ex}{0ex}
\titlespacing{\subsubsection}{0ex}{0.5ex}{0ex}

% Pengaturan format potongan kode
\usepackage{listings}
\definecolor{comment}{RGB}{0,128,0}
\definecolor{string}{RGB}{255,0,0}
\definecolor{keyword}{RGB}{0,0,255}
\lstdefinestyle{codestyle}{
  commentstyle=\color{comment},
  stringstyle=\color{string},
  keywordstyle=\color{keyword},
  basicstyle=\footnotesize\ttfamily,
  numbers=left,
  numberstyle=\tiny,
  numbersep=5pt,
  frame=lines,
  breaklines=true,
  prebreak=\raisebox{0ex}[0ex][0ex]{\ensuremath{\hookleftarrow}},
  showstringspaces=false,
  upquote=true,
  tabsize=2,
}
\lstset{style=codestyle}

% Isi keseluruhan dokumen
\begin{document}

  % Sampul luar Bahasa Indonesia
  \newcommand\covercontents{sampul/konten-id.tex}
  \AddToShipoutPictureBG*{
  \AtPageLowerLeft{
    % Ubah nilai berikut jika posisi horizontal background tidak sesuai
    \hspace{-3.5mm}

    % Ubah nilai berikut jika posisi vertikal background tidak sesuai
    \raisebox{0mm}{
      \includegraphics[width=\paperwidth,height=\paperheight]{sampul/sampul-luar.png}
    }
  }
}

% Menyembunyikan nomor halaman
\thispagestyle{empty}

% Pengaturan margin untuk menyesuaikan konten sampul
\newgeometry{top=95mm,left=25mm,right=20mm,bottom=25mm}

\begin{flushleft}

  % Pengaturan jenis dan warna teks yang digunakan
  \sffamily\color{white}

  % Ubah penomoran buku berikut sesuai dengan yang ditentukan oleh departemen
  \noindent\textbf{TUGAS AKHIR - TD123456}
  \vspace{6ex}

  % Ubah kalimat berikut sesuai dengan judul tugas akhir
  \noindent{\large \textbf{KALKULASI ENERGI PADA ROKET LUAR ANGKASA BERBASIS \emph{ANTI GRAVITASI}}}
  \vspace{4ex}

  % Ubah kalimat berikut sesuai dengan nama mahasiswa
  \noindent\textbf{Elon Reeve Musk} \\
  % Ubah kalimat berikut sesuai dengan NRP mahasiswa
  \textbf{NRP 0123 20 4000 0001}
  \vspace{2ex}

  \noindent\textbf{Dosen Pembimbing} \\
  % Ubah kalimat berikut sesuai dengan nama-nama dosen pembimbing
  \textbf{Nikola Tesla, S.T., M.T.} \\
  \textbf{Wernher von Braun, S.T., M.T.}
  \vspace{6ex}

  % Ubah kalimat berikut sesuai dengan nama departemen
  \noindent\textbf{DEPARTEMEN TEKNIK DIRGANTARA} \\
  % Ubah kalimat berikut sesuai dengan nama fakultas
  \textbf{Fakultas Teknologi Dirgantara} \\
  \textbf{Institut Teknologi Sepuluh Nopember} \\
  % Ubah kalimat berikut sesuai dengan tempat dan tahun pembuatan buku
  \textbf{Surabaya 2021}

\end{flushleft}

\restoregeometry

  % Atur ulang penomoran halaman
  \setcounter{page}{1}

  % Sampul dalam Bahasa Indonesia
  \renewcommand\covercontents{sampul/konten-id.tex}
  \AddToShipoutPictureBG*{
  \AtPageLowerLeft{
    % Ubah nilai berikut jika posisi horizontal background tidak sesuai
    \hspace{-3.5mm}

    % Ubah nilai berikut jika posisi vertikal background tidak sesuai
    \raisebox{0mm}{
      \includegraphics[width=\paperwidth,height=\paperheight]{sampul/gambar/sampul-dalam.png}
    }
  }
}

% Menyembunyikan nomor halaman
\thispagestyle{empty}

% Pengaturan margin untuk menyesuaikan konten sampul
\newgeometry{
  top=95mm,
  left=25mm,
  right=20mm,
  bottom=25mm
}

\begin{flushleft}

  % Pemilihan font sans serif
  \sffamily

  % Pemilihan font bold
  \fontseries{bx}
  \selectfont

  \input{\covercontents}

\end{flushleft}

\restoregeometry

  \clearpage
  \cleardoublepage

  % Sampul dalam Bahasa Inggris
  \renewcommand\covercontents{sampul/konten-en.tex}
  \AddToShipoutPictureBG*{
  \AtPageLowerLeft{
    % Ubah nilai berikut jika posisi horizontal background tidak sesuai
    \hspace{-3.5mm}

    % Ubah nilai berikut jika posisi vertikal background tidak sesuai
    \raisebox{0mm}{
      \includegraphics[width=\paperwidth,height=\paperheight]{sampul/gambar/sampul-dalam.png}
    }
  }
}

% Menyembunyikan nomor halaman
\thispagestyle{empty}

% Pengaturan margin untuk menyesuaikan konten sampul
\newgeometry{
  top=95mm,
  left=25mm,
  right=20mm,
  bottom=25mm
}

\begin{flushleft}

  % Pemilihan font sans serif
  \sffamily

  % Pemilihan font bold
  \fontseries{bx}
  \selectfont

  \input{\covercontents}

\end{flushleft}

\restoregeometry

  \cleardoublepage

  % Pengaturan ukuran indentasi paragraf
  \setlength{\parindent}{2em}

  % Pengaturan ukuran spasi paragraf
  \setlength{\parskip}{1ex}

  % Pernyataan keaslian
  \begin{center}
  \large
  \textbf{PERNYATAAN ORISINALITAS}
\end{center}

% Menyembunyikan nomor halaman
\thispagestyle{empty}

\vspace{2ex}

% Ubah paragraf-paragraf berikut sesuai dengan yang ingin diisi pada pernyataan keaslian

\noindent Yang bertanda tangan dibawah ini:

\noindent\begin{tabularx}{\textwidth}{l l X}
                         &   &                            \\
  Nama Mahasiswa / NRP   & : & \name{} / \nrp{}           \\
  Departemen             & : & \department{}              \\
  Dosen Pembimbing / NIP & : & \advisor{} / \advisornip{} \\
                         &   &                            \\
\end{tabularx}

Dengan ini menyatakan bahwa Tugas Akhir dengan judul "\tatitle{}" adalah hasil karya sendiri, berfsifat orisinal, dan ditulis dengan mengikuti kaidah penulisan ilmiah.

Bilamana di kemudian hari ditemukan ketidaksesuaian dengan pernyataan ini, maka saya bersedia menerima sanksi sesuai dengan ketentuan yang berlaku di Institut Teknologi Sepuluh Nopember.

\vspace{8ex}

\noindent\begin{tabularx}{\textwidth}{X l}
                     & \place{}, \ENGMONTH{} \the\year{} \\
                     &                                   \\
  Mengetahui         &                                   \\
  Dosen Pembimbing   & Mahasiswa                         \\
                     &                                   \\
                     &                                   \\
                     &                                   \\
                     &                                   \\
                     &                                   \\
  \advisor{}         & \name{}                           \\
  NIP. \advisornip{} & NRP. \nrp{}                       \\
\end{tabularx}

  \cleardoublepage

  % Lembar pengesahan
  \begin{center}
  \large
  \textbf{LEMBAR PENGESAHAN}
\end{center}

% Menyembunyikan nomor halaman
\thispagestyle{empty}

\begin{center}
  % Ubah kalimat berikut dengan judul tugas akhir
  \textbf{KALKULASI ENERGI PADA ROKET LUAR ANGKASA BERBASIS \emph{ANTI-GRAVITASI}}
\end{center}

\begingroup
% Pemilihan font ukuran small
\small

\begin{center}
  \textbf{TUGAS AKHIR}
  \\Diajukan untuk memenuhi salah satu syarat memperoleh gelar Sarjana Teknik pada Program Studi S-1 Teknik Komputer Departemen Teknik Komputer Fakultas Teknologi Elektro dan Informatika Cerdas Institut Teknologi Sepuluh Nopember
\end{center}

\begin{center}
  Oleh: \name{}
  \\NRP. \nrp{}
\end{center}

\begin{center}
  Disetujui oleh Tim Penguji Tugas Akhir:
\end{center}

\begingroup
% Menghilangkan padding
\setlength{\tabcolsep}{0pt}

\noindent
\begin{tabularx}{\textwidth}{X l}
  % Ubah kalimat-kalimat berikut dengan nama dosen pembimbing pertama
  \advisor{}                       & (Pembimbing I)                      \\
  NIP: \advisornip{}               &                                     \\
                                   & ................................... \\
                                   &                                     \\
                                   &                                     \\
  % Ubah kalimat-kalimat berikut dengan nama dosen pembimbing kedua
  \coadvisor{}                     & (Pembimbing II)                     \\
  NIP: \coadvisornip{}             &                                     \\
                                   & ................................... \\
                                   &                                     \\
                                   &                                     \\
  % Ubah kalimat-kalimat berikut dengan nama dosen penguji pertama
  Dr. Galileo Galilei, S.T., M.Sc. & (Penguji I)                         \\
  NIP: 18560710 194301 1 001       &                                     \\
                                   & ................................... \\
                                   &                                     \\
                                   &                                     \\
  % Ubah kalimat-kalimat berikut dengan nama dosen penguji kedua
  Friedrich Nietzsche, S.T., M.Sc. & (Penguji II)                        \\
  NIP: 18560710 194301 1 001       &                                     \\
                                   & ................................... \\
                                   &                                     \\
                                   &                                     \\
  % Ubah kalimat-kalimat berikut dengan nama dosen penguji ketiga
  Alan Turing, ST., MT.            & (Penguji III)                       \\
  NIP: 18560710 194301 1 001       &                                     \\
                                   & ................................... \\
                                   &                                     \\
                                   &                                     \\
\end{tabularx}
\endgroup

\begin{center}
  % Ubah kalimat berikut dengan jabatan kepala departemen
  Mengetahui, \\
  Kepala Departemen \department{} \facultyshort{} - ITS\\

  \vspace{8ex}

  % Ubah kalimat-kalimat berikut dengan nama dan NIP kepala departemen
  \underline{Dr. Supeno Mardi Susiki Nugroho, S.T., M.T.} \\
  NIP. 19700313 199512 1 001
\end{center}

\begin{center}
  \textbf{\MakeUppercase{\place{}}\\Bulan, \the\year{}}
\end{center}
\endgroup

  \cleardoublepage

  % Nomor halaman pembuka dimulai dari sini
  \pagenumbering{roman}

  % Abstrak Bahasa Indonesia
  \begin{center}
  \large\textbf{ABSTRAK}
\end{center}

\vspace{2ex}

\begingroup
  % Menghilangkan padding
  \setlength{\tabcolsep}{0pt}

  \noindent
  \begin{tabularx}{\textwidth}{l >{\centering}m{3em} X}
    % Ubah kalimat-kalimat berikut sesuai dengan nama dan NRP mahasiswa
    Nama Mahasiswa  &:& Elon Reeve Musk \\
    NRP             &:&	0123 20 4000 0001 \\

    % Ubah kalimat-kalimat berikut sesuai dengan nama-nama dosen pembimbing
    Pembimbing      &:& 1. Nikola Tesla, S.T., M.T. \\
                    & & 2. Wernher von Braun, S.T., M.T. \\
  \end{tabularx}
\endgroup

% Ubah paragraf berikut sesuai dengan abstrak dari tugas akhir.
Pada penelitian ini kami mengajukan \lipsum[1]

% Ubah kata-kata berikut sesuai dengan kata kunci dari tugas akhir.
Kata Kunci: Roket, \emph{Anti-gravitasi}, Energi, Angkasa.

  \cleardoublepage

  % Abstrak Bahasa Inggris
  \begin{center}
  \large\textbf{ABSTRACT}
\end{center}

\addcontentsline{toc}{chapter}{ABSTRACT}

\vspace{2ex}

\begingroup
% Menghilangkan padding
\setlength{\tabcolsep}{0pt}

\noindent
\begin{tabularx}{\textwidth}{l >{\centering}m{3em} X}
  \emph{Name}     & : & \name{}                                                             \\

  \emph{Title}    & : & \engtatitle{} \\

  \emph{Advisors} & : & 1. \advisor{}                                                       \\
                  &   & 2. \coadvisor{}                                                     \\
\end{tabularx}
\endgroup

% Ubah paragraf berikut dengan abstrak dari tugas akhir dalam Bahasa Inggris
\emph{In this research, we proposed \lipsum[1]}

% Ubah kata-kata berikut dengan kata kunci dari tugas akhir dalam Bahasa Inggris
\emph{Keywords}: \emph{Rocket}, \emph{Anti-gravity}, \emph{Energy}, \emph{Space}.

  \cleardoublepage

  % Kata pengantar
  \begin{center}
  \Large
  \textbf{KATA PENGANTAR}
\end{center}

\addcontentsline{toc}{chapter}{KATA PENGANTAR}

\vspace{2ex}

% Ubah paragraf-paragraf berikut dengan isi dari kata pengantar

Puji dan syukur kehadirat \lipsum[1][1-5]

Penelitian ini disusun dalam rangka \lipsum[2][1-5]
Oleh karena itu, penulis mengucapkan terima kasih kepada:

\begin{enumerate}[nolistsep]

  \item Keluarga, Ibu, Bapak dan Saudara tercinta yang telah \lipsum[3][1-2]

  \item Bapak Nikola Tesla, S.T., M.T., selaku \lipsum[4][1-2]

  \item \lipsum[5][1-3]

\end{enumerate}

Akhir kata, semoga \lipsum[6][1-8]

\begin{flushright}
  \begin{tabular}[b]{c}
    % Ubah kalimat berikut dengan tempat, bulan, dan tahun penulisan
    Surabaya, Mei 2021\\
    \\
    \\
    \\
    \\
    % Ubah kalimat berikut dengan nama mahasiswa
    Elon Reeve Musk
  \end{tabular}
\end{flushright}

  \cleardoublepage

  % Daftar isi
  \renewcommand*\contentsname{DAFTAR ISI}
  \addcontentsline{toc}{chapter}{\contentsname}
  \tableofcontents
  \cleardoublepage

  % Daftar gambar
  \renewcommand*\listfigurename{DAFTAR GAMBAR}
  \addcontentsline{toc}{chapter}{\listfigurename}
  \listoffigures
  \cleardoublepage

  % Daftar tabel
  \renewcommand*\listtablename{DAFTAR TABEL}
  \addcontentsline{toc}{chapter}{\listtablename}
  \listoftables
  \cleardoublepage

  % Nomor halaman isi dimulai dari sini
  \pagenumbering{arabic}

  % Bab 1 pendahuluan
  \chapter{PENDAHULUAN}
\label{chap:pendahuluan}

% Ubah bagian-bagian berikut dengan isi dari pendahuluan

Penelitian ini dilatarbelakangi oleh \lipsum[1][1-5]

\section{Latar Belakang}
\label{sec:latarbelakang}

Pesatnya perkembangan roket yang merupakan \lipsum[1]

\lipsum[2]

\section{Permasalahan}
\label{sec:permasalahan}

Dari permasalahan tersebut maka \lipsum[1][1-6]

\section{Tujuan}
\label{sec:Tujuan}

Tujuan dari \lipsum[1][1-3] adalah:

\begin{enumerate}[nolistsep]

  \item Membuat \lipsum[2][1-3]

  \item \lipsum[3][1-3]

\end{enumerate}

\section{Batasan Masalah}
\label{sec:batasanmasalah}

Batasan-batasan dari \lipsum[1][1-3] adalah:

\begin{enumerate}[nolistsep]

  \item Mempermudah \lipsum[2][1-3]

  \item \lipsum[3][1-5]

  \item \lipsum[4][1-5]

\end{enumerate}

\section{Sistematika Penulisan}
\label{sec:sistematikapenulisan}

Laporan penelitian tugas akhir ini terbagi menjadi \lipsum[1][1-3] yaitu:

\begin{enumerate}[nolistsep]

  \item \textbf{BAB I Pendahuluan}

  Bab ini berisi \lipsum[2][1-5]

  \vspace{2ex}

  \item \textbf{BAB II Tinjauan Pustaka}

  Bab ini berisi \lipsum[3][1-5]

  \vspace{2ex}

  \item \textbf{BAB III Desain dan Implementasi Sistem}

  Bab ini berisi \lipsum[4][1-5]

  \vspace{2ex}

  \item \textbf{BAB IV Pengujian dan Analisa}

  Bab ini berisi \lipsum[5][1-5]

  \vspace{2ex}

  \item \textbf{BAB V Penutup}

  Bab ini berisi \lipsum[6][1-5]

\end{enumerate}

  \cleardoublepage

  % Bab 2 tinjauan pustaka
  \chapter{TINJAUAN PUSTAKA}
\label{chap:tinjauanpustaka}

% Ubah bagian-bagian berikut dengan isi dari tinjauan pustaka

Demi mendukung penelitian ini, \lipsum[1][1-5]

\section{Roket Luar Angkasa}
\label{sec:roketluarangkasa}

% Contoh input gambar
\begin{figure}[ht]
  \centering

  % Ubah dengan nama file gambar dan ukuran yang akan digunakan
  \includegraphics[scale=0.35]{gambar/roketluarangkasa.jpg}

  % Ubah dengan keterangan gambar yang diinginkan
  \caption{Peluncuran roket luar angkasa \emph{Discovery} \citep{roketluarangkasa}.}
  \label{fig:roketluarangkasa}
\end{figure}

Roket luar angkasa merupakan \lipsum[1]

\emph{Discovery}, Gambar \ref{fig:roketluarangkasa}, merupakan \lipsum[2]

\section{Gravitasi}
\label{sec:gravitasi}

Gravitasi merupakan \lipsum[1]

\subsection{Hukum Newton}
\label{subsec:hukumnewton}

Newton \citep{newton1687} pernah merumuskan bahwa \lipsum[1]
Kemudian menjadi persamaan seperti pada persamaan \ref{eq:hukumpertamanewton}.

% Contoh pembuatan persamaan
\begin{equation}
  \label{eq:hukumpertamanewton}
  \sum \mathbf{F} = 0\; \Leftrightarrow\; \frac{\mathrm{d} \mathbf{v} }{\mathrm{d}t} = 0.
\end{equation}

\subsection{Anti Gravitasi}
\label{subsec:antigravitasi}

Anti gravitasi merupakan \lipsum[1]

  \cleardoublepage

  % Bab 3 desain dan implementasi
  \chapter{DESAIN DAN IMPLEMENTASI}
\label{chap:desainimplementasi}

% Ubah bagian-bagian berikut dengan isi dari desain dan implementasi

Penelitian ini dilaksanakan sesuai \lipsum[1][1-5]

\section{Deskripsi Sistem}
\label{sec:deskripsisistem}

Sistem akan dibuat dengan \lipsum[1-2]

\section{Implementasi Alat
\label{sec:implementasi alat}}

Alat diimplementasikan dengan \lipsum[1]

% Contoh pembuatan potongan kode
\begin{lstlisting}[
  language=C++,
  caption={Program halo dunia.},
  label={lst:halodunia}
]
#include <iostream>

int main() {
    std::cout << "Halo Dunia!";
    return 0;
}
\end{lstlisting}

\lipsum[2-3]

% Contoh input potongan kode dari file
\lstinputlisting[
  language=Python,
  caption={Program perhitungan bilangan prima.},
  label={lst:bilanganprima}
]{program/bilangan-prima.py}

\lipsum[4]

  \cleardoublepage

  % Bab 4 pengujian dan analisis
  \chapter{PENGUJIAN DAN ANALISIS}
\label{chap:pengujiananalisis}

% Ubah bagian-bagian berikut dengan isi dari pengujian dan analisis

Pada penelitian ini dipaparkan \lipsum[1][1-5]

\section{Skenario Pengujian}
\label{sec:skenariopengujian}

Pengujian dilakukan dengan \lipsum[1-2]

\section{Evaluasi Pengujian}
\label{sec:analisispengujian}

Dari pengujian yang \lipsum[1]

% Contoh pembuatan tabel
\begin{longtable}{|c|c|c|}
  \caption{Hasil Pengukuran Energi dan Kecepatan}
  \label{tb:EnergiKecepatan}\\
  \hline
  \rowcolor[HTML]{C0C0C0}
  \textbf{Energi} & \textbf{Jarak Tempuh} & \textbf{Kecepatan} \\
  \hline
  10 J & 1000 M & 200 M/s \\
  20 J & 2000 M & 400 M/s \\
  30 J & 4000 M & 800 M/s \\
  40 J & 8000 M & 1600 M/s \\
  \hline
\end{longtable}

\lipsum[2-4]

  \cleardoublepage

  % Bab 5 penutup
  \chapter{PENUTUP}
\label{chap:penutup}

% Ubah bagian-bagian berikut dengan isi dari penutup

\section{Kesimpulan}
\label{sec:kesimpulan}

Berdasarkan hasil pengujian yang \lipsum[1][1-3] sebagai berikut:

\begin{enumerate}[nolistsep]

  \item Pembuatan \lipsum[2][1-3]

  \item \lipsum[2][4-6]

  \item \lipsum[2][7-10]

\end{enumerate}

\section{Saran}
\label{chap:saran}

Untuk pengembangan lebih lanjut pada \lipsum[1][1-3] antara lain:

\begin{enumerate}[nolistsep]

  \item Memperbaiki \lipsum[2][1-3]

  \item \lipsum[2][4-6]

  \item \lipsum[2][7-10]

\end{enumerate}

  \cleardoublepage

  % Daftar pustaka
  \renewcommand\bibname{DAFTAR PUSTAKA}
  \addcontentsline{toc}{chapter}{\bibname}
  \bibliographystyle{unsrtnat}
  \bibliography{pustaka/pustaka.bib}
  \cleardoublepage

  % Biografi penulis
  \begin{center}
  \Large
  \textbf{BIOGRAFI PENULIS}
\end{center}

\addcontentsline{toc}{chapter}{BIOGRAFI PENULIS}

\vspace{2ex}

\begin{wrapfigure}{L}{0.3\textwidth}
  \centering
  \vspace{-3ex}
  % Ubah file gambar berikut dengan file foto dari mahasiswa
  \includegraphics[width=0.3\textwidth]{gambar/elon.jpg}
  \vspace{-4ex}
\end{wrapfigure}

% Ubah kalimat berikut dengan biografi dari mahasiswa
Elon Reeve Musk, lahir pada \lipsum[1]

\lipsum[2]

  \cleardoublepage

\end{document}
